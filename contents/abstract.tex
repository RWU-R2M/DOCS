\section{Abstract}

In the Canadian International Rover Challenge (CIRC), university student teams from various countries are tasked with creating prototype rovers to simulate early colony activities on another planet. These rovers must complete a series of tasks, such as navigating different terrains, performing autonomous operations, and using a dexterous arm to manipulate objects. This challenge highlights practical solutions for future extraterrestrial exploration.

\vspace{5mm} %VERTICAL SPACE

The CIRC team at RWU is divided into 3 separate departments, being "Mechanical Engineering", "Information Technology" and "Power Electronics". Those sub-departments face their very own tasks and play a key role in their field of working and research, being equally important as a whole for developing a rover from scratch. A typical competition rover weighs around 50kg and comes with a footprint of up to several meters in length. This is necessary to carry the payload and equipment determined by the aimed functionality of the rover, as well as maneuvering in the difficult desert-like environment. 

\vspace{5mm} %VERTICAL SPACE

In the process of building and working with the rover, possible hazards and failure-scenarios must be kept in mind. It is especially necessary to consider all likelihoods of electrical risks, as these are the most common and devastating ones. Electric shocks, release of toxic materials and fumes, burn injuries, fire or even explosions are possible threats to name a few of them. Comprehensive planning is not only adviced but necessary in this case, also to guarantee a harmonious interaction of the whole system. At the time of writing, the general electrical layout has been designed and the construction phase is ongoing. Not all components have been chosen yet, neither implemented in hardware. Due to this reason, further investigation and work is needed in that regard and details about individual circuitry can't be included in this paper (as of July 2024). 
